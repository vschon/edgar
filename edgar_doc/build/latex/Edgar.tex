% Generated by Sphinx.
\def\sphinxdocclass{report}
\documentclass[letterpaper,10pt,english]{sphinxmanual}
\usepackage[utf8]{inputenc}
\DeclareUnicodeCharacter{00A0}{\nobreakspace}
\usepackage{cmap}
\usepackage[T1]{fontenc}
\usepackage{babel}
\usepackage{times}
\usepackage[Bjarne]{fncychap}
\usepackage{longtable}
\usepackage{sphinx}
\usepackage{multirow}


\title{Edgar Documentation}
\date{December 09, 2013}
\release{0.1.1}
\author{Brandon}
\newcommand{\sphinxlogo}{}
\renewcommand{\releasename}{Release}
\makeindex

\makeatletter
\def\PYG@reset{\let\PYG@it=\relax \let\PYG@bf=\relax%
    \let\PYG@ul=\relax \let\PYG@tc=\relax%
    \let\PYG@bc=\relax \let\PYG@ff=\relax}
\def\PYG@tok#1{\csname PYG@tok@#1\endcsname}
\def\PYG@toks#1+{\ifx\relax#1\empty\else%
    \PYG@tok{#1}\expandafter\PYG@toks\fi}
\def\PYG@do#1{\PYG@bc{\PYG@tc{\PYG@ul{%
    \PYG@it{\PYG@bf{\PYG@ff{#1}}}}}}}
\def\PYG#1#2{\PYG@reset\PYG@toks#1+\relax+\PYG@do{#2}}

\def\PYG@tok@gd{\def\PYG@tc##1{\textcolor[rgb]{0.63,0.00,0.00}{##1}}}
\def\PYG@tok@gu{\let\PYG@bf=\textbf\def\PYG@tc##1{\textcolor[rgb]{0.50,0.00,0.50}{##1}}}
\def\PYG@tok@gt{\def\PYG@tc##1{\textcolor[rgb]{0.00,0.25,0.82}{##1}}}
\def\PYG@tok@gs{\let\PYG@bf=\textbf}
\def\PYG@tok@gr{\def\PYG@tc##1{\textcolor[rgb]{1.00,0.00,0.00}{##1}}}
\def\PYG@tok@cm{\let\PYG@it=\textit\def\PYG@tc##1{\textcolor[rgb]{0.25,0.50,0.56}{##1}}}
\def\PYG@tok@vg{\def\PYG@tc##1{\textcolor[rgb]{0.73,0.38,0.84}{##1}}}
\def\PYG@tok@m{\def\PYG@tc##1{\textcolor[rgb]{0.13,0.50,0.31}{##1}}}
\def\PYG@tok@mh{\def\PYG@tc##1{\textcolor[rgb]{0.13,0.50,0.31}{##1}}}
\def\PYG@tok@cs{\def\PYG@tc##1{\textcolor[rgb]{0.25,0.50,0.56}{##1}}\def\PYG@bc##1{\colorbox[rgb]{1.00,0.94,0.94}{##1}}}
\def\PYG@tok@ge{\let\PYG@it=\textit}
\def\PYG@tok@vc{\def\PYG@tc##1{\textcolor[rgb]{0.73,0.38,0.84}{##1}}}
\def\PYG@tok@il{\def\PYG@tc##1{\textcolor[rgb]{0.13,0.50,0.31}{##1}}}
\def\PYG@tok@go{\def\PYG@tc##1{\textcolor[rgb]{0.19,0.19,0.19}{##1}}}
\def\PYG@tok@cp{\def\PYG@tc##1{\textcolor[rgb]{0.00,0.44,0.13}{##1}}}
\def\PYG@tok@gi{\def\PYG@tc##1{\textcolor[rgb]{0.00,0.63,0.00}{##1}}}
\def\PYG@tok@gh{\let\PYG@bf=\textbf\def\PYG@tc##1{\textcolor[rgb]{0.00,0.00,0.50}{##1}}}
\def\PYG@tok@ni{\let\PYG@bf=\textbf\def\PYG@tc##1{\textcolor[rgb]{0.84,0.33,0.22}{##1}}}
\def\PYG@tok@nl{\let\PYG@bf=\textbf\def\PYG@tc##1{\textcolor[rgb]{0.00,0.13,0.44}{##1}}}
\def\PYG@tok@nn{\let\PYG@bf=\textbf\def\PYG@tc##1{\textcolor[rgb]{0.05,0.52,0.71}{##1}}}
\def\PYG@tok@no{\def\PYG@tc##1{\textcolor[rgb]{0.38,0.68,0.84}{##1}}}
\def\PYG@tok@na{\def\PYG@tc##1{\textcolor[rgb]{0.25,0.44,0.63}{##1}}}
\def\PYG@tok@nb{\def\PYG@tc##1{\textcolor[rgb]{0.00,0.44,0.13}{##1}}}
\def\PYG@tok@nc{\let\PYG@bf=\textbf\def\PYG@tc##1{\textcolor[rgb]{0.05,0.52,0.71}{##1}}}
\def\PYG@tok@nd{\let\PYG@bf=\textbf\def\PYG@tc##1{\textcolor[rgb]{0.33,0.33,0.33}{##1}}}
\def\PYG@tok@ne{\def\PYG@tc##1{\textcolor[rgb]{0.00,0.44,0.13}{##1}}}
\def\PYG@tok@nf{\def\PYG@tc##1{\textcolor[rgb]{0.02,0.16,0.49}{##1}}}
\def\PYG@tok@si{\let\PYG@it=\textit\def\PYG@tc##1{\textcolor[rgb]{0.44,0.63,0.82}{##1}}}
\def\PYG@tok@s2{\def\PYG@tc##1{\textcolor[rgb]{0.25,0.44,0.63}{##1}}}
\def\PYG@tok@vi{\def\PYG@tc##1{\textcolor[rgb]{0.73,0.38,0.84}{##1}}}
\def\PYG@tok@nt{\let\PYG@bf=\textbf\def\PYG@tc##1{\textcolor[rgb]{0.02,0.16,0.45}{##1}}}
\def\PYG@tok@nv{\def\PYG@tc##1{\textcolor[rgb]{0.73,0.38,0.84}{##1}}}
\def\PYG@tok@s1{\def\PYG@tc##1{\textcolor[rgb]{0.25,0.44,0.63}{##1}}}
\def\PYG@tok@gp{\let\PYG@bf=\textbf\def\PYG@tc##1{\textcolor[rgb]{0.78,0.36,0.04}{##1}}}
\def\PYG@tok@sh{\def\PYG@tc##1{\textcolor[rgb]{0.25,0.44,0.63}{##1}}}
\def\PYG@tok@ow{\let\PYG@bf=\textbf\def\PYG@tc##1{\textcolor[rgb]{0.00,0.44,0.13}{##1}}}
\def\PYG@tok@sx{\def\PYG@tc##1{\textcolor[rgb]{0.78,0.36,0.04}{##1}}}
\def\PYG@tok@bp{\def\PYG@tc##1{\textcolor[rgb]{0.00,0.44,0.13}{##1}}}
\def\PYG@tok@c1{\let\PYG@it=\textit\def\PYG@tc##1{\textcolor[rgb]{0.25,0.50,0.56}{##1}}}
\def\PYG@tok@kc{\let\PYG@bf=\textbf\def\PYG@tc##1{\textcolor[rgb]{0.00,0.44,0.13}{##1}}}
\def\PYG@tok@c{\let\PYG@it=\textit\def\PYG@tc##1{\textcolor[rgb]{0.25,0.50,0.56}{##1}}}
\def\PYG@tok@mf{\def\PYG@tc##1{\textcolor[rgb]{0.13,0.50,0.31}{##1}}}
\def\PYG@tok@err{\def\PYG@bc##1{\fcolorbox[rgb]{1.00,0.00,0.00}{1,1,1}{##1}}}
\def\PYG@tok@kd{\let\PYG@bf=\textbf\def\PYG@tc##1{\textcolor[rgb]{0.00,0.44,0.13}{##1}}}
\def\PYG@tok@ss{\def\PYG@tc##1{\textcolor[rgb]{0.32,0.47,0.09}{##1}}}
\def\PYG@tok@sr{\def\PYG@tc##1{\textcolor[rgb]{0.14,0.33,0.53}{##1}}}
\def\PYG@tok@mo{\def\PYG@tc##1{\textcolor[rgb]{0.13,0.50,0.31}{##1}}}
\def\PYG@tok@mi{\def\PYG@tc##1{\textcolor[rgb]{0.13,0.50,0.31}{##1}}}
\def\PYG@tok@kn{\let\PYG@bf=\textbf\def\PYG@tc##1{\textcolor[rgb]{0.00,0.44,0.13}{##1}}}
\def\PYG@tok@o{\def\PYG@tc##1{\textcolor[rgb]{0.40,0.40,0.40}{##1}}}
\def\PYG@tok@kr{\let\PYG@bf=\textbf\def\PYG@tc##1{\textcolor[rgb]{0.00,0.44,0.13}{##1}}}
\def\PYG@tok@s{\def\PYG@tc##1{\textcolor[rgb]{0.25,0.44,0.63}{##1}}}
\def\PYG@tok@kp{\def\PYG@tc##1{\textcolor[rgb]{0.00,0.44,0.13}{##1}}}
\def\PYG@tok@w{\def\PYG@tc##1{\textcolor[rgb]{0.73,0.73,0.73}{##1}}}
\def\PYG@tok@kt{\def\PYG@tc##1{\textcolor[rgb]{0.56,0.13,0.00}{##1}}}
\def\PYG@tok@sc{\def\PYG@tc##1{\textcolor[rgb]{0.25,0.44,0.63}{##1}}}
\def\PYG@tok@sb{\def\PYG@tc##1{\textcolor[rgb]{0.25,0.44,0.63}{##1}}}
\def\PYG@tok@k{\let\PYG@bf=\textbf\def\PYG@tc##1{\textcolor[rgb]{0.00,0.44,0.13}{##1}}}
\def\PYG@tok@se{\let\PYG@bf=\textbf\def\PYG@tc##1{\textcolor[rgb]{0.25,0.44,0.63}{##1}}}
\def\PYG@tok@sd{\let\PYG@it=\textit\def\PYG@tc##1{\textcolor[rgb]{0.25,0.44,0.63}{##1}}}

\def\PYGZbs{\char`\\}
\def\PYGZus{\char`\_}
\def\PYGZob{\char`\{}
\def\PYGZcb{\char`\}}
\def\PYGZca{\char`\^}
\def\PYGZsh{\char`\#}
\def\PYGZpc{\char`\%}
\def\PYGZdl{\char`\$}
\def\PYGZti{\char`\~}
% for compatibility with earlier versions
\def\PYGZat{@}
\def\PYGZlb{[}
\def\PYGZrb{]}
\makeatother

\begin{document}

\maketitle
\tableofcontents
\phantomsection\label{index::doc}


Contents:


\chapter{Module: Form10DB}
\label{form10DB:welcome-to-edgar-s-documentation}\label{form10DB::doc}\label{form10DB:module-form10DB}\label{form10DB:module-form10db}\index{form10DB (module)}
edgar.form10DB includes functions to download 10 forms from dataset.

It includes ticker to cik transformation, quarter master file download, and 10 form download functions.
\index{CIKTranslator (class in form10DB)}

\begin{fulllineitems}
\phantomsection\label{form10DB:form10DB.CIKTranslator}\pysigline{\strong{class }\code{form10DB.}\bfcode{CIKTranslator}}
CIKTranslator manages the mapping between cik and ticker
\begin{description}
\item[{Users can get the dictionary from two members:}] \leavevmode
CIKtranslator.tickerGetter:   dictionary with cik:ticker

CIKtranslator.cikGetter:      dictionary with ticker:cik

\end{description}
\index{updateCIKtable() (form10DB.CIKTranslator method)}

\begin{fulllineitems}
\phantomsection\label{form10DB:form10DB.CIKTranslator.updateCIKtable}\pysiglinewithargsret{\bfcode{updateCIKtable}}{\emph{tickers}}{}
generate a txt file that map ticker to cik
\begin{description}
\item[{Args:}] \leavevmode
sequence of symbol(str): symbols to add to the dictionary

\item[{Return:}] \leavevmode
emptycik(list): list of ticker without cik

\item[{Output:}] \leavevmode
DATA\_ADD + `/Edgar/cik2ticker.txt

\end{description}

\end{fulllineitems}


\end{fulllineitems}

\index{Form10Manager (class in form10DB)}

\begin{fulllineitems}
\phantomsection\label{form10DB:form10DB.Form10Manager}\pysigline{\strong{class }\code{form10DB.}\bfcode{Form10Manager}}
MasterFileManager download and maintains quarterly master file of
EDGAR database
\index{buildIndex() (form10DB.Form10Manager method)}

\begin{fulllineitems}
\phantomsection\label{form10DB:form10DB.Form10Manager.buildIndex}\pysiglinewithargsret{\bfcode{buildIndex}}{}{}
build index file for each ticker directory

\end{fulllineitems}

\index{listQuraterMasterFile() (form10DB.Form10Manager method)}

\begin{fulllineitems}
\phantomsection\label{form10DB:form10DB.Form10Manager.listQuraterMasterFile}\pysiglinewithargsret{\bfcode{listQuraterMasterFile}}{}{}
List all available quarterly master file

\end{fulllineitems}

\index{updateAndDownload() (form10DB.Form10Manager method)}

\begin{fulllineitems}
\phantomsection\label{form10DB:form10DB.Form10Manager.updateAndDownload}\pysiglinewithargsret{\bfcode{updateAndDownload}}{\emph{year}, \emph{qtr}, \emph{range='all'}}{}
update master file and download new forms

\end{fulllineitems}

\index{updateForms() (form10DB.Form10Manager method)}

\begin{fulllineitems}
\phantomsection\label{form10DB:form10DB.Form10Manager.updateForms}\pysiglinewithargsret{\bfcode{updateForms}}{\emph{year}, \emph{qtr}, \emph{range='all'}}{}
download all 10 forms listed in the master file to corresponding ticker directory
\begin{description}
\item[{Args:}] \leavevmode
year,qtr(int):  year and quarter of master file
range(str):     specify date range to download
\begin{quote}

`all': download all forms in the master file
`today': download forms with date == today
\end{quote}

\end{description}

return:

\end{fulllineitems}

\index{updateQuarterMasterFile() (form10DB.Form10Manager method)}

\begin{fulllineitems}
\phantomsection\label{form10DB:form10DB.Form10Manager.updateQuarterMasterFile}\pysiglinewithargsret{\bfcode{updateQuarterMasterFile}}{\emph{beginyear}, \emph{beginqtr}, \emph{endyear}, \emph{endqtr}}{}
Download quarterly master file from edgar database
Args:
\begin{quote}

beginyear, ..., endqtr (int): specify required date range
\end{quote}

\end{fulllineitems}


\end{fulllineitems}



\chapter{Module: forecaster}
\label{forecaster:module-forecaster}\label{forecaster::doc}\index{forecaster (module)}\index{DictionaryGenerator (class in forecaster)}

\begin{fulllineitems}
\phantomsection\label{forecaster:forecaster.DictionaryGenerator}\pysigline{\strong{class }\code{forecaster.}\bfcode{DictionaryGenerator}}
Generate a dictionary from the files indicated in the keyfile
\index{GenerateDictionary() (forecaster.DictionaryGenerator method)}

\begin{fulllineitems}
\phantomsection\label{forecaster:forecaster.DictionaryGenerator.GenerateDictionary}\pysiglinewithargsret{\bfcode{GenerateDictionary}}{\emph{keyfile}, \emph{outdictname}}{}~\begin{description}
\item[{Args:}] \leavevmode
keyfile(str): path to look for the keyfile
outdictname: name of the output dictionary

\end{description}

\end{fulllineitems}


\end{fulllineitems}

\index{Forecaster (class in forecaster)}

\begin{fulllineitems}
\phantomsection\label{forecaster:forecaster.Forecaster}\pysigline{\strong{class }\code{forecaster.}\bfcode{Forecaster}}
Forecaster is wrapper class for different classifiers
\begin{description}
\item[{Forecaster can:}] \leavevmode\begin{enumerate}
\item {} 
train and test on historical edgar data

\item {} 
make predictions on new incoming data

\end{enumerate}

\item[{Constructor Args:}] \leavevmode
indexFilePath(str): path to the index file of all training data

\end{description}
\index{GenerateTrainTestDateSet() (forecaster.Forecaster method)}

\begin{fulllineitems}
\phantomsection\label{forecaster:forecaster.Forecaster.GenerateTrainTestDateSet}\pysiglinewithargsret{\bfcode{GenerateTrainTestDateSet}}{\emph{traincut}, \emph{devcut}}{}
cut training set and testing set based on cut date(cut date include in training set)

\end{fulllineitems}

\index{setClassifier() (forecaster.Forecaster method)}

\begin{fulllineitems}
\phantomsection\label{forecaster:forecaster.Forecaster.setClassifier}\pysiglinewithargsret{\bfcode{setClassifier}}{\emph{classifier\_}}{}
set the classifier of the forecaster.

The classifier should have a fit(X,y), score(X,y) and predict(X) functions
\begin{description}
\item[{Args:}] \leavevmode
classifier: instance of a classifier

\end{description}

\end{fulllineitems}


\end{fulllineitems}

\index{doMostCommonClassify() (in module forecaster)}

\begin{fulllineitems}
\phantomsection\label{forecaster:forecaster.doMostCommonClassify}\pysiglinewithargsret{\code{forecaster.}\bfcode{doMostCommonClassify}}{\emph{keyfile}, \emph{outfilename}}{}
guess all label to be 1, 2,3
Most Common Baseline

\end{fulllineitems}

\index{doWordListClassify() (in module forecaster)}

\begin{fulllineitems}
\phantomsection\label{forecaster:forecaster.doWordListClassify}\pysiglinewithargsret{\code{forecaster.}\bfcode{doWordListClassify}}{\emph{keyfile}, \emph{outfilename}, \emph{sue\_weight=0.0}, \emph{k=0}}{}
count pos/neg words to make classification

\end{fulllineitems}



\chapter{Module: utils}
\label{utils:module-utils}\label{utils::doc}\label{utils:module-form10DB}\index{form10DB (module)}
edgar.form10DB includes functions to download 10 forms from dataset.

It includes ticker to cik transformation, quarter master file download, and 10 form download functions.
\index{CIKTranslator (class in form10DB)}

\begin{fulllineitems}
\phantomsection\label{utils:form10DB.CIKTranslator}\pysigline{\strong{class }\code{form10DB.}\bfcode{CIKTranslator}}
CIKTranslator manages the mapping between cik and ticker
\begin{description}
\item[{Users can get the dictionary from two members:}] \leavevmode
CIKtranslator.tickerGetter:   dictionary with cik:ticker

CIKtranslator.cikGetter:      dictionary with ticker:cik

\end{description}
\index{updateCIKtable() (form10DB.CIKTranslator method)}

\begin{fulllineitems}
\phantomsection\label{utils:form10DB.CIKTranslator.updateCIKtable}\pysiglinewithargsret{\bfcode{updateCIKtable}}{\emph{tickers}}{}
generate a txt file that map ticker to cik
\begin{description}
\item[{Args:}] \leavevmode
sequence of symbol(str): symbols to add to the dictionary

\item[{Return:}] \leavevmode
emptycik(list): list of ticker without cik

\item[{Output:}] \leavevmode
DATA\_ADD + `/Edgar/cik2ticker.txt

\end{description}

\end{fulllineitems}


\end{fulllineitems}

\index{Form10Manager (class in form10DB)}

\begin{fulllineitems}
\phantomsection\label{utils:form10DB.Form10Manager}\pysigline{\strong{class }\code{form10DB.}\bfcode{Form10Manager}}
MasterFileManager download and maintains quarterly master file of
EDGAR database
\index{buildIndex() (form10DB.Form10Manager method)}

\begin{fulllineitems}
\phantomsection\label{utils:form10DB.Form10Manager.buildIndex}\pysiglinewithargsret{\bfcode{buildIndex}}{}{}
build index file for each ticker directory

\end{fulllineitems}

\index{listQuraterMasterFile() (form10DB.Form10Manager method)}

\begin{fulllineitems}
\phantomsection\label{utils:form10DB.Form10Manager.listQuraterMasterFile}\pysiglinewithargsret{\bfcode{listQuraterMasterFile}}{}{}
List all available quarterly master file

\end{fulllineitems}

\index{updateAndDownload() (form10DB.Form10Manager method)}

\begin{fulllineitems}
\phantomsection\label{utils:form10DB.Form10Manager.updateAndDownload}\pysiglinewithargsret{\bfcode{updateAndDownload}}{\emph{year}, \emph{qtr}, \emph{range='all'}}{}
update master file and download new forms

\end{fulllineitems}

\index{updateForms() (form10DB.Form10Manager method)}

\begin{fulllineitems}
\phantomsection\label{utils:form10DB.Form10Manager.updateForms}\pysiglinewithargsret{\bfcode{updateForms}}{\emph{year}, \emph{qtr}, \emph{range='all'}}{}
download all 10 forms listed in the master file to corresponding ticker directory
\begin{description}
\item[{Args:}] \leavevmode
year,qtr(int):  year and quarter of master file
range(str):     specify date range to download
\begin{quote}

`all': download all forms in the master file
`today': download forms with date == today
\end{quote}

\end{description}

return:

\end{fulllineitems}

\index{updateQuarterMasterFile() (form10DB.Form10Manager method)}

\begin{fulllineitems}
\phantomsection\label{utils:form10DB.Form10Manager.updateQuarterMasterFile}\pysiglinewithargsret{\bfcode{updateQuarterMasterFile}}{\emph{beginyear}, \emph{beginqtr}, \emph{endyear}, \emph{endqtr}}{}
Download quarterly master file from edgar database
Args:
\begin{quote}

beginyear, ..., endqtr (int): specify required date range
\end{quote}

\end{fulllineitems}


\end{fulllineitems}



\chapter{Module: DataGenerator}
\label{DataGenerator:module-DataGenerator}\label{DataGenerator:module-datagenerator}\label{DataGenerator::doc}\index{DataGenerator (module)}\index{DatabaseIndex() (in module DataGenerator)}

\begin{fulllineitems}
\phantomsection\label{DataGenerator:DataGenerator.DatabaseIndex}\pysiglinewithargsret{\code{DataGenerator.}\bfcode{DatabaseIndex}}{\emph{sector}}{}
generate a key file indexing all data points for a sector
file format:
\begin{quote}

date    symbol  label   path
\end{quote}
\begin{description}
\item[{TODO:}] \leavevmode
reset path

\end{description}

\end{fulllineitems}

\index{GenerateDataset() (in module DataGenerator)}

\begin{fulllineitems}
\phantomsection\label{DataGenerator:DataGenerator.GenerateDataset}\pysiglinewithargsret{\code{DataGenerator.}\bfcode{GenerateDataset}}{\emph{sector}, \emph{data}, \emph{begin}, \emph{end}}{}
Generate data set
Args:
\begin{quote}

sector(str): currenyly only `Industrials' available
data: data returned by loadData
begin: begin index of the symbol list
end:    end index of the symbol list
\end{quote}

\end{fulllineitems}

\index{checkEmpty() (in module DataGenerator)}

\begin{fulllineitems}
\phantomsection\label{DataGenerator:DataGenerator.checkEmpty}\pysiglinewithargsret{\code{DataGenerator.}\bfcode{checkEmpty}}{\emph{symbolList}}{}
return ticker list with empty index file
\begin{description}
\item[{Args:}] \leavevmode
symbolList: list of symbols

\item[{Return:}] \leavevmode\begin{description}
\item[{result:\{`NoSymbol': symbols not in the cik-ticker dict}] \leavevmode
`NoIndex': symbol in the cik-ticker dict, but does not have index file\}

\end{description}

\end{description}

\end{fulllineitems}

\index{cleanPrice() (in module DataGenerator)}

\begin{fulllineitems}
\phantomsection\label{DataGenerator:DataGenerator.cleanPrice}\pysiglinewithargsret{\code{DataGenerator.}\bfcode{cleanPrice}}{\emph{price}}{}
check same ticker used by two companies

\end{fulllineitems}

\index{loadData() (in module DataGenerator)}

\begin{fulllineitems}
\phantomsection\label{DataGenerator:DataGenerator.loadData}\pysiglinewithargsret{\code{DataGenerator.}\bfcode{loadData}}{\emph{sector}}{}
read the pricce, portfolio and sue data into memory

\end{fulllineitems}

\index{updateCoreIndex() (in module DataGenerator)}

\begin{fulllineitems}
\phantomsection\label{DataGenerator:DataGenerator.updateCoreIndex}\pysiglinewithargsret{\code{DataGenerator.}\bfcode{updateCoreIndex}}{}{}
Core index file stores 10QK filing information of all ticker
Update core index file based on the indexfile of each stock

\end{fulllineitems}



\chapter{Indices and tables}
\label{index:indices-and-tables}\begin{itemize}
\item {} 
\emph{genindex}

\item {} 
\emph{modindex}

\item {} 
\emph{search}

\end{itemize}


\renewcommand{\indexname}{Python Module Index}
\begin{theindex}
\def\bigletter#1{{\Large\sffamily#1}\nopagebreak\vspace{1mm}}
\bigletter{d}
\item {\texttt{DataGenerator}}, \pageref{DataGenerator:module-DataGenerator}
\indexspace
\bigletter{f}
\item {\texttt{forecaster}}, \pageref{forecaster:module-forecaster}
\item {\texttt{form10DB}}, \pageref{utils:module-form10DB}
\end{theindex}

\renewcommand{\indexname}{Index}
\printindex
\end{document}
